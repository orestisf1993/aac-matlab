\section{Level 1}
\subsection{Υλοποίηση}
\subsubsection{Συναρτήσεις για ελέγχους}\label{sub:asserts1}
Οι συναρτήσεις με όνομα \texttt{assert*} χρησιμοποιούνται σε διάφορα μέρη στο project για τη πραγματοποίηση απλών ελέγχων των δεδομένων.
Στο πρώτο επίπεδο χρησιμοποιούνται οι εξής:
\begin{enumerate}
\item \texttt{assertIsFrameType}
\item \texttt{assertIsFullFrame}
\item \texttt{assertIsWinType}
\end{enumerate}

\subsubsection{\texttt{SSC}}
\begin{itemize}
\item Για το φιλτράρισμα των frame με το υψιπερατό φίλτρο χρησιμοποιείται η υποσυνάρτηση \texttt{filterFrame()} που χρησιμοποιεί τη συνάρτηση \mintinline{MATLAB}!filter! του MATLAB.
\item Η λογική επιλογής του frame $i$ σύμφωνα με το frame $i-1$ γίνεται με τις υποσυναρτήσεις \texttt{nextFrameIsESH()} και \texttt{typeFromAdjacent()}.
\item Η λογική για την απόφαση του τελικού τύπου frame σύμφωνα με το πίνακα 1 της εκφώνησης γίνεται με τη δημιουργία map στην υποσυνάρτηση \texttt{decisionTable()}.
\end{itemize}

\subsubsection{\texttt{filterbank} και \texttt{iFilterbank}}
\begin{itemize}
\item Η δημιουργία των παραθύρων γίνεται στη συνάρτηση \texttt{createWindow()}.

Για την δημιουργία παραθύρων τύπου \texttt{KDB} \textbf{δεν} χρησιμοποιήθηκε η συνάρτηση \mintinline{MATLAB}!kaiser! του MATLAB αλλά η \mintinline{MATLAB}!besseli! για την υλοποίηση σύμφωνα με τους τύπους της εκφώνησης.
Τα παράθυρα KDB ελέγχτηκαν μέσω της συνθήκης Princen-Bradley.
Μέρος της συνάρτησης μεταφράστηκε από το αρχείο \texttt{kbdwin.c} από \url{http://www.ee.columbia.edu/~marios/mdct/mdct_giraffe.html}.

Η δημιουργία παραθύρων SIN είναι απλή.

\item Η υποσυνάρτηση \texttt{mdct()} και η συνάρτηση \texttt{imdct()} βρέθηκαν στο \url{http://www.ee.columbia.edu/~marios/mdct/mdct_giraffe.html} και υποστήκαν μικρές αλλαγές.
\end{itemize}

\subsubsection{\texttt{AACoder1} και \texttt{iAACoder1}}\label{sub:AACoder1}
Η \texttt{AACoder1}, αφού διαβάσει το αρχείο \texttt{fNameIn}, προσθέτει από μισό frame με μηδενικά στην αρχή και στο τέλος της ακολουθίας \texttt{input} που πρόκειται να κωδικοποιηθεί.
Αυτό βοηθάει στη μείωση του θορύβου που εμφανίζεται στα άκρα της ακολουθίας μετά την κωδικοποίηση και προκαλεί μείωση του $SNR$.
Η \texttt{iAACoder1} αναλαμβάνει την αφαίρεση αυτών των ημι-frame μετά την αποκωδικοποίηση.

Η κωδικοποίηση γίνεται με μια λούπα \mintinline{MATLAB}!for! για κάθε ένα από τα \texttt{numberOfFrames} frames.
Η διαδικασία για κάθε frame ακολουθείται σε σχέση με το διάγραμμα της εκφώνησης.
Η τιμή \texttt{prevType} θεωρείται ότι είναι πάντα \mintinline{MATLAB}!'OLS'! για το πρώτο frame.

\subsubsection{\texttt{demoAAC1}}
Για να αποφευχθεί η εγγραφή του ίδιου κώδικα πολλές φορές και τα 3 "demo" του project χρησιμοποιούν τη κοινή συνάρτηση \texttt{genericDemo()}\label{genericDemo}.
Σε αυτή επιλέγονται η συνάρτηση κωδικοποίησης \texttt{encodeFun} και η συνάρτηση αποκωδικοποίησης \texttt{decodeFun} σύμφωνα με το όρισμα \texttt{level} (τιμές 1, 2 ή 3).
Έτσι, η συνάρτηση \texttt{demoAAC1} αρκεί να την καλέσει ως \mintinline{MATLAB}!SNR = genericDemo(fNameIn, fNameOut, 1)!.

Το signal to noise ratio υπολογίζεται μέσω της συνάρτησης του MATLAB \mintinline{MATLAB}!snr!.

\begin{itemize}
\item Η ακολουθία \texttt{input} διαβάζεται μέσω της συνάρτησης του MATLAB \mintinline{MATLAB}!audioread!.

\item Η \texttt{output} υπολογίζεται μετά την αποκωδικοποίηση:
\mintinline{MATLAB}!output = decodeFun(AACSeq, fNameOut)!.

\item Το \texttt{SNR} υπολογίζεται ως:
\begin{minted}{MATLAB}
common_length = min(length(input), length(output));
input = input(1:common_length,:);
output = output(1:common_length,:);
noise = input - output;

SNR = snr(input, noise);
\end{minted}
\end{itemize}

\subsection{Αποτελέσματα}\label{sub:results1}
Παρατηρούμε γρήγορη εκτέλεση και signal to noise ratio ίσο με $307 dB$.
Αυτό σημαίνει ότι οι μόνες απώλειες που έχουμε είναι λόγω της ακρίβειας των πράξεων του MATLAB.
\begin{code}
\begin{minted}{MATLAB}
>> SNR = demoAAC1('input.wav', 'output.wav')
Encoding:Elapsed time is 0.957438 seconds.
Decoding:Elapsed time is 0.225739 seconds.
Level 1: SNR for channel 1: 307.89.
Level 1: SNR for channel 2: 307.942.

SNR =

  307.9169
\end{minted}
\caption{SNR για την υλοποίηση του πρώτου επιπέδου με παράθυρο \texttt{SIN}}
\end{code}
