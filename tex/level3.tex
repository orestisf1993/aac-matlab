\section{Level 3}
\subsection{Υλοποίηση}
\subsubsection{\texttt{psycho}}
\begin{itemize}
\item Η spreading function υπολογίζεται σε κάθε κλήση της \texttt{psycho} μέσω της υποσυνάρτησης \texttt{spreadingFunction()} για κάθε πιθανό συνδυασμό των τιμών index μπάντας:
\begin{minted}{MATLAB}
for b1 = bb
    for b2 = bb
        spreading(b1, b2) = spreadingFunction(b1, b2, bval);
    end
end
\end{minted}

\item Το παράθυρο Hann υπολογίζεται σύμφωνα με τον τύπο:
\begin{minted}{MATLAB}
hannWindow = 0.5 - 0.5 * cos(pi / N * ((0:N - 1) + 0.5));
hannWindow = hannWindow(:);
\end{minted}

\item Η υποσυνάρτηση \texttt{frameFFT()} υπολογίζει των FFT ενός frame $2048 \times 1$ και υπολογίζει σε μορφή μέτρου και ορίσματος τα πρώτα $1024$ στοιχεία του στα \texttt{r} και \texttt{f}.

\item Τα υπόλοιπα βήματα ακολουθούνται όπως αναγράφονται στην εκφώνηση με πράξεις διανυσμάτων ή με λούπες \mintinline{MATLAB}!for!.
\end{itemize}

\subsubsection{\texttt{AACquantizer} και \texttt{iAACquantizer}}
\begin{itemize}
\item Η ενέργεια των συντελεστών MDCT, τόσο για τον υπολογισμό των \texttt{T} όσο και για τον υπολογισμό της ενέργειας σφάλματος κβαντισμού, υπολογίζεται μέσω της υποσυνάρτησης \texttt{bandEnergy()}.

\item Για τον υπολογισμό των \texttt{a}, αφού αρχικοποιηθούν στη τιμή \mintinline{MATLAB}!floor(16/3*log2(max(frame)^(3 / 4)/MQ))!, χρησιμοποιείται η επανάληψη:
\begin{minted}{MATLAB}
while true
    a = aNew;
    P = quantizationError(frame, a, bands);
    aNew = a + (P < T);
    if all(aNew == a) || max(abs(diff(aNew))) > 60
        break
    end
end
\end{minted}
Η επανάληψη "σπάει" όταν όλες οι νέες τιμές \texttt{aNew} είναι ίσες με τις παλιές τιμές \texttt{a} ή όταν φτάσουμε τη μέγιστη διαφορά μεταξύ δύο στοιχείων του \texttt{a}.

\item Η υποσυνάρτηση \texttt{quantizationError()} καλεί την υποσυνάρτηση \texttt{quantize()} πάνω στο frame και τη συνάρτηση \texttt{deQuantize()} πάνω στο αποτέλεσμα της προηγούμενης:
\begin{minted}{MATLAB}
a = bandStretch(a, size(X), bands);
S = quantize(X, a);
Xbar = deQuantize(S, a);
P = bandEnergy(X-Xbar, bands);
\end{minted}

\item Η \texttt{bandStretch} χρησιμοποιείται για την μετατροπή ενός διανύσματος μεγέθους $N_B \times 1$ σε $1024 \times 1$.

\item Οι διαφορές των συντελεστών υπολογίζονται μέσω της συνάρτησης του MATLAB \mintinline{MATLAB}!diff! ενώ η αντίστροφη διαδικασία στην \texttt{iAACquantizer()} γίνεται μέσω της συνάρτησης του MATLAB \mintinline{MATLAB}!cumsum!.
\end{itemize}

\subsubsection{\texttt{AACoder3} και \texttt{iAACoder3}}
Σε σχέση με τα προηγούμενα επίπεδα πραγματοποιήθηκαν οι εξής αλλαγές:
\begin{itemize}
\item Αρχικοποιούνται 2 μηδενικά frames \texttt{frameTprev1} και \texttt{frameTprev2} για τη χρήση στη \texttt{psycho()}.
Αυτά, στη συνέχεια, κρατάνε τις τιμές των προηγούμενων frame.

\item Για κάθε κανάλι ξεχωριστά καλείται η \texttt{psycho} στο \texttt{frameT}, η \texttt{AACquantizer} και οι \texttt{encodeHuff} και \texttt{decodeHuff} (που μας δίνονται).
Τα ορίσματα στην \texttt{encodeHuff} περνιούνται σαν μονοδιάστατα διανύσματα ενώ τα αποτελέσματα της \texttt{decodeHuff} μετατρέπονται στο κατάλληλο μέγεθος σύμφωνα με το \texttt{frameType}.

\item Αν δοθεί το όρισμα \texttt{fnameAACoded} αποθηκεύεται το \mintinline{MATLAB}!struct AACSeq3!.

\item Η ακολουθία \texttt{decoded} στη \texttt{iAACoder3} κανονικοποιείται ως:
\begin{minted}{MATLAB}
decoded(:, 1) = decoded(:, 1) ./ max(abs(decoded(:, 1)));
decoded(:, 2) = decoded(:, 2) ./ max(abs(decoded(:, 2)));
\end{minted}
καθώς η \mintinline{MATLAB}!audiowrite! αποθηκεύει δεδομένα που βρίσκονται μεταξύ των τιμών \texttt{0} και \texttt{1}.
\end{itemize}

\subsubsection{\texttt{demoAAC3}}
Η συνάρτηση \texttt{demoAAC3} αρκεί να καλέσει την \hyperref[genericDemo]{\texttt{genericDemo}}
ως \mintinline{MATLAB}![SNR, bitrate, compression] = genericDemo(fNameIn, fNameOut, 3, fnameAACoded)!.

Το μέγεθος του αρχικού αρχείου ήχου υπολογίζεται μέσω της εντολής \mintinline{MATLAB}!dir! του MATLAB από το αρχείο \texttt{fNameIn}.

Για το μέγεθος του συμπιεσμένου αποτελέσματος έγινε η εξής παρατήρηση:
Σύμφωνα με το documentation του MATLAB η εντολή \mintinline{MATLAB}!save! έκδοσης \texttt{7} και μεγαλύτερης πραγματοποιεί αυτόματη συμπίεση των δεδομένων.
Επίσης, ενώ τα αποτελέσματα της \texttt{encodeHuff} είναι μια σειρά δυαδικών αριθμών, το MATLAB αποθηκεύει μη αποδοτικά τα δεδομένα, σαν σειρά από \mintinline{MATLAB}!char!.
Για αυτό, το "πραγμάτικο" μέγεθος των συμπιεσμένων δεδομένων γίνεται να υπολογιστεί είτε μέσω της εντολής \mintinline{MATLAB}!dir! από το αρχείο \texttt{fnameAACoded} ή με εναλλακτικό τρόπο.
Η επιλογή της μεθόδου μέτρησης γίνεται με το flag \texttt{useDirOnCompressed}.

Ο εναλλακτικός τρόπος είναι ο εξής:
\begin{minted}{MATLAB}
for idx = 1:length(AACSeq)
    compressedSizeBits = compressedSizeBits ...
        + numel(AACSeq(idx).chl.sfc) ...
        + numel(AACSeq(idx).chr.sfc) ...
        + numel(AACSeq(idx).chl.stream) ...
        + numel(AACSeq(idx).chr.stream) ...
        + (numel(AACSeq(idx).chl.TNScoeffs) + 1) * bitsPerDouble ...
        + (numel(AACSeq(idx).chr.TNScoeffs) + 1) * bitsPerDouble ...
        + numel(AACSeq(idx).chl.codebook) * bitsPerInt ...
        + numel(AACSeq(idx).chr.codebook) * bitsPerInt ...
        + 2 + 1;
end
\end{minted}
Για κάθε μέλος του \mintinline{MATLAB}!struct AACSeq!:
\begin{itemize}
\item Μετράμε το κάθε ψηφίο των ακολουθιών \texttt{sfc} από κάθε channel ως 1 bit.
\item Μετράμε το κάθε ψηφίο των ακολουθιών \texttt{stream} από κάθε channel ως 1 bit.
\item Μετράμε το κάθε στοιχείο των πινάκων \texttt{TNScoeffs} από κάθε channel έως 64 bit σαν double precision floating point.
\item Προσθέτουμε το μέγεθος 2 integers για τα \texttt{codebook}, 2 bits για την αποθήκευση του \texttt{frameType} και 1 bit για το \texttt{winType}.
\end{itemize}
Φυσικά, αυτή η μέτρηση είναι πρόχειρη και το πραγματικό μέγεθος σε πραγματική εφαρμογή μπορεί να ήταν διαφορετικό.

Θεωρούνται $8$ bits ανά byte και $1000$ bytes ανά kilobyte.

\subsection{Αποτελέσματα}
Φυσικά, λόγω της κωδικοποίησης και της απώλειας πληροφορίας, το signal to noise ratio έπεσε σημαντικά.
Ωστόσο, στο αυτί το τραγούδι δεν φαίνεται να έχει μεγάλη απώλειας.

\begin{code}
\begin{minted}{MATLAB}
>> [SNR, bitrate, compression] = demoAAC3('input.wav', 'output.wav', 'level3.mat')
Encoding:Saving encoding in file level3.mat.
Elapsed time is 122.143561 seconds.
Decoding:Elapsed time is 3.155757 seconds.
Level 3: SNR for channel 1: 5.79205.
Level 3: SNR for channel 2: 5.91146.
Uncompressed audio: 1131956 bytes (9055648 bits).
Compressed struct : 244628 bytes (1957024 bits).
Compression ratio : 21.611087% (x 4.627254).
Bitrate: 333.583636 kbits per second.

SNR =

    5.8533


bitrate =

   3.3358e+05


compression =

    0.2161
\end{minted}
\caption{SNR για την υλοποίηση του τρίτου επιπέδου με παράθυρο \texttt{SIN} και με μετρήσεις με την εντολή \texttt{dir}}
\end{code}

\begin{code}
\begin{minted}{MATLAB}
>> [SNR, bitrate, compression] = demoAAC3('input.wav', 'output.wav', 'level3.mat')
Encoding:Saving encoding in file level3.mat.
Elapsed time is 121.206160 seconds.
Decoding:Elapsed time is 3.083816 seconds.
Level 3: SNR for channel 1: 5.79205.
Level 3: SNR for channel 2: 5.91146.
Uncompressed audio: 1131956 bytes (9055648 bits).
Compressed struct : 11044832 bytes (1380604 bits).
Compression ratio : 15.245778% (x 6.559193).
Bitrate: 235.330227 kbits per second.

SNR =

    5.8533


bitrate =

   2.3533e+05


compression =

    0.1525
\end{minted}
\caption{SNR για την υλοποίηση του τρίτου επιπέδου με παράθυρο \texttt{SIN} και με τον εναλλακτικό τρόπο μέτρησης μεγέθους}
\end{code}
